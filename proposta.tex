\documentclass[conference]{IEEEtran}
\usepackage{filecontents}
\usepackage[noadjust]{cite}
\usepackage[portuguese]{babel}
\usepackage[utf8]{inputenc}

\begin{filecontents*}{bibi.bib}
	
     @ARTICLE{lee2005tarboard,
   author = {Lee, Wonwoo and Woo, Woontack and Lee, Jongweon},
   title = {Tarboard: Tangible augmented reality system for table-top game environment},
   journal = {2nd International Workshop on Pervasive Gaming Applications, PerGamese},
   year = {2005},
   volume = {5},
   pages = {0-4}
   }
   
   @book{burdea2003virtual,
   	title={Virtual Reality Technology},
   	author={Burdea, G.C. and Coiffet, P.},
   	number={v. 1},
   	isbn={9780471360896},
   	lccn={2002038088},
   	series={Academic Search Complete},
   	year={2003},
   	publisher={Wiley}
   }
   
   @inproceedings{Lam:2006:AAR:1128923.1128987,
   	author = {Lam, Albert H. T. and Chow, Kevin C. H. and Yau, Edward H. H. and Lyu, Michael R.},
   	title = {ART: Augmented Reality Table for Interactive Trading Card Game},
   	booktitle = {Proceedings of the 2006 ACM International Conference on Virtual Reality Continuum and Its Applications},
   	series = {VRCIA '06},
   	year = {2006},
   	isbn = {1-59593-324-7},
   	location = {Hong Kong, China},
   	pages = {357--360},
   	numpages = {4},
   	doi = {10.1145/1128923.1128987},
   	acmid = {1128987},
   	publisher = {ACM},
   	address = {New York, NY, USA},
   	keywords = {augmented reality, card game environment, computer entertainment},
   } 
   
   @inproceedings{Szalavari:1998:CGA:293701.293740,
   	author = {Szalav\'{a}ri, Zsolt and Eckstein, Erik and Gervautz, Michael},
   	title = {Collaborative Gaming in Augmented Reality},
   	booktitle = {Proceedings of the ACM Symposium on Virtual Reality Software and Technology},
   	series = {VRST '98},
   	year = {1998},
   	isbn = {1-58113-019-8},
   	location = {Taipei, Taiwan},
   	pages = {195--204},
   	numpages = {10},
   	doi = {10.1145/293701.293740},
   	acmid = {293740},
   	publisher = {ACM},
   	address = {New York, NY, USA},
   	keywords = {CSCW, augmented reality, interaction, virtual gaming},
   } 
   
   @misc{magic,
   	title = {Magic: The Gathering},
   	howpublished = {\url{http://magic.wizards.com/pt-br}},
   	note = {Accessed: 2015-09-15}
   }
   
   @misc{magic_duels,
   	title = {Magic Duels},
   	howpublished = {\url{http://magic.wizards.com/content/magic-duels}},
   	note = {Accessed: 2015-09-15}
   }
   
   @misc{unity,
   	title = {Unity - Game Engine},
   	howpublished = {\url{http://unity3d.com/pt/}},
   	note = {Accessed: 2015-09-16}
   }
   
   @misc{vuforia,
   	title = {Vuforia Developer Portal},
   	howpublished = {\url{https://developer.vuforia.com/}},
   	note = {Accessed: 2015-09-16}
   }
   
   @misc{csharp,
   	title = "Guia de Programação em {C}{\#}",
   	howpublished = {\url{https://msdn.microsoft.com/pt-br/library/67ef8sbd.aspx}},
   	note = {Accessed: 2015-09-16}
   }
   
   @misc{magic_cards,
   	title = "MagicCards.info",
   	howpublished = {\url{http://magiccards.info/}},
   	note = {Accessed: 2015-09-16}
   }
\end{filecontents*}

\hyphenation{op-tical net-works semi-conduc-tor}    

\title{Jogo de Cartas Remoto em Realidade Aumentada}

\author{Vítor de Albuquerque Torreão}


\markboth{Disciplina de Realidade Virtual, Setembro~2015}%
{Shell \MakeLowercase{\textit{et al.}}: Bare Demo of IEEEtran.cls for Journals}

\begin{document}
\maketitle

\begin{abstract}
	Os jogos do mundo real e virtual possuem suas próprias vantagens distintas. 
	A realidade aumentada nos permite combinar o melhor dos dois mundos e criar 
	novas formas de jogar. Esta proposta pretende introduzir o ARCGP (Plataforma 
	para jogos de cartas com realidade aumentada). A ideia é prover aos 
	jogadores a experiência dos jogos originais, porém com a possibilidade de 
	jogar remotamente, sem ter de sentar à frente do computador e com baixo 
	custo, sem necessidade de câmeras, mesas vidradas, espelho ou televisores. 
	A plataforma requer apenas um dispositivo capaz de rodar uma aplicação 
	Unity, seja um smartphone ou computador pessoal.
\end{abstract}

\begin{IEEEkeywords}
	Realidade Aumentada, Entretenimento.
\end{IEEEkeywords}

\section{Introdução}
Realidade misturada combina os conteúdos do mundo real com a imaginação virtual.
A realidade aumentada (AR) é um subconjunto da realidade misturada, onde o 
conteúdo digital são sobrepostos aos objetos reais do mundo. Aplicações de 
realidade aumentada suplementam o mundo real com objetos virtuais, de forma que 
o conteúdo gerado pelo computador é adicionado ao ambiente real de forma 
interativa e em tempo real \cite{burdea2003virtual}.

Essas propriedades da realidade aumentada provêm melhorias fascinantes para os 
jogos do mundo real, fazendo-os mais agradáveis e atrativos 
\cite{Lam:2006:AAR:1128923.1128987}. Jogos de cartas são exemplos de jogos que 
podem usufruir das vantagens da Realidade Aumentada. Jogos, como \textit{Magic: 
The Gathering} \cite{magic}, precisam de, no mínimo, dois jogadores. Mas não é 
sempre fácil reunir-se com outras pessoas para jogar, de forma que uma 
plataforma que permitisse a dois jogadores disputar uma partida de forma 
remota criaria mais oportunidades de jogo.

Já houveram implementações de \textit{Magic} para computador e dispositivos 
móveis \cite{magic_duels}. No entanto, as plataformas nos quais essas 
aplicações estão disponíveis quebram  a familiaridade do jogador com a forma 
de jogar com a qual está acostumado.

A plataforma hipotética deveria então, manter elementos das partidas de forma 
que o jogador tivesse a impressão de estar jogando o mesmo jogo, da mesma forma.

Neste artigo será proposta a plataforma ARCGP (\textit{Augmented Reality Card 
Game Platform}) para jogos de cartas. A plataforma consistirá em uma aplicação 
Unity \cite{unity} utilizando a API do Vuforia \cite{vuforia}. 

Este artigo está estruturado da seguinte forma: na sessão \ref{objetivo}, serão 
expostos os objetivos da plataforma, tal como foi idealizada; na sessão 
\ref{rel} serão apresentados trabalhos relacionados; na sessão \ref{metodologia} 
será apresentada a metodologia a ser utilizada no desenvolvimento da plataforma; 
na sessão \ref{resultados_esperados}, serão apontados os resultados esperados 
neste projeto; na sessão \ref{cronograma} será descrito o cronograma a ser 
seguido para o desenvolvimento; por fim, a sessão \ref{conclusao} trás algumas 
considerações finais sobre o projeto.

\section{Objetivo}
\label{objetivo}
O objetivo do projeto ARCGP é utilizar o Unity e o Vuforia criar uma plataforma 
de realidade aumentada onde dois jogadores possam disputar uma partida de um 
jogo de cartas sem a necessidade de estarem no mesmo ambiente. ARCGP deve 
reconhecer as cartas em jogo no tabuleiro de um jogador e reproduzi-las através 
de objetos virtuais para o oponente.

O ARCGP deve ser uma plataforma simples, que ocupe pouco espaço físico, que não 
apresente um esforço desnecessário de montagem e que seja de baixo custo.

Como o Vuforia não consegue rastrear mais do que cinco alvos de imagem, serão 
utilizados marcadores. Os jogadores terão de registrar suas cartas previamentem, 
momento no qual será feito mapeamento entre as cartas e os identificadores de 
marcadores disponíveis no Vuforia.

Uma imagem será selecionada como alvo para um \textit{extended tracking} 
(rastreamento extendido, em português) e será colocada por cada jogador na sua 
mesa de jogo.

A aplicação deverá ser capaz de se conectar ao dispositivo do adversário. E, a 
partir de então, deverá rastrear os marcadores e exibir para o jogador a imagem 
da carta mapeana no identificador do marcador em questão. Quando este marcador 
for colocado sobre a mesa de jogo, o ARCGP deverá exibir a imagem da carta para 
o jogador, ao mesmo tempo que envia os dados da carta e sua posição em relação 
ao marcador da mesa para o dispositivo do adversário, que irá, por sua vez, 
transladar e rotacionar a carta para exibí-la corretamente.

\section{Trabalhos Relacionados}
\label{rel}
Os trabalhos realizados envolvendo jogos de cartas com realidade aumentada 
trabalham em cima do cenário onde dois jogadores estão no mesmo ambiente e focam 
trazer a imaginação do jogador para a realidade através da sobreposição de 
objetos virtuais. As plataformas propostas, no entanto, requerem um hardware 
específico e, muitas vezes, de alto custo.

Em \cite{Szalavari:1998:CGA:293701.293740}, os pesquisadores montaram um sistema 
de realidade aumentada para jogadores compartilharem um ambiente virtual onde 
jogam \textit{Mah-Jongg}. Os objetivos do sistema desenvolvido por eles é 
aplicar realidade aumentada para melhorar a experiência do usuário mantendo os 
aspectos da interação social e privacidade. O sistema consiste de um 
\textit{head mounted display} (HMD) e um painel de interação pessoal para cada 
jogador, além de um servidor que roda a simulação do jogo. O HMD não bloqueia 
totalmente a visão do ambiente, permitindo a um jogador se comunicar com o outro 
através de gestos, e os elementos virtuais exibidos no HMD permitem que cada 
usuário tenho uma visão diferente do jogo, garantindo a privacidade.

Assim, \cite{Szalavari:1998:CGA:293701.293740} apresenta um sistema com uma 
aplicação diferente da plataforma proposta neste artigo já que ele visa simular 
um jogo de mesa e exibi-lo aos jogadores por meio de realidade aumentada, 
enquanto que neste projeto a realidade aumentada é utilizada para permitir que 
jogadores distantes possam jogar como se estivessem no mesmo ambiente, sem 
necessidade de simular o jogo em computador.

O TARBoard \cite{lee2005tarboard} é um sistema que utiliza a realidade aumentada 
e a interface de usuário tangível para melhorar a experiência do usuário nos 
jogos de cartas e de tabuleiro em geral. O TARBoard consiste de uma mesa de 
vidro, duas câmeras e um espelho. Marcadores são colocados embaixo dos objetos 
e/ou cartas do jogo. Uma das câmeras fica posicionada embaixo da mesa e rastreia 
os marcadores. A segunda câmera provê imagens para aumentar os objetos virtuais. 
Os usuários veem os objetos virtuais no \textit{stream} de vídeo das câmeras.

O objetivo do TARBoard é bem diferente do ARCGP, pois ele pretende melhorar a 
experiência tradicional do jogo sem, no entanto, substituí-la como o ARCGP.

O ART (\textit{Augmented Reality Table}) \cite{Lam:2006:AAR:1128923.1128987} é 
uma plataforma que usa a tecnologia da realidade aumentada para prover uma mesa 
virtual para os jogadores. Ela consiste de um computador, um monitor e uma 
câmera. A câmara, que é o único dispositivo de entrada do sistema, fica 
posicionada acima dos jogadores e captura os eventos da mesa. O monitor fica 
posicionado na horizontal servindo como mesa para os jogadores, que jogam de 
forma natural e tradicional. Este monitor exibe para os jogadores o ambiente 
virtual com som e imagem aumentados. O computador é a unidade de processamento 
do sistema, capturando as imagens, reconhecendo as entradas válidas e produzindo 
as saídas correspondentes de acordo com as regras do jogo.

É importante observar algumas limitações do ART. Primeiramente, é necessário 
que os jogadores estejam no mesmo ambiente. Além disso, é necessário montar 
toda uma estrutura de hardware para o correto funcionamento da plataforma: 
posicionar a câmera de forma a capturar os eventos da mesa, posicionar o monitor 
corretamente e conectar ambos ao computador. Por fim, esses equipamentos 
possuem um custo elevado para os jogadores.

\section{Metodologia}
\label{metodologia}
Para desenvolver essa plataforma será utilizado o jogo \textit{Magic: The 
Gathering} como exemplo. 

Primeiramente, será necessário um estudo aprofundado do Unity, da lingaugem de 
programação C\# \cite{csharp} e do Vuforia para, depois, montar uma arquitetura 
viável para a aplicação.

Em seguida, será necessário criar um banco de dados para testes com algumas 
cartas. Existem diversos \textit{websites} que disponibilizam imagens de cartas, 
como por exemplo, \cite{magic_cards}.

Depois, será escolhida uma imagem que servirá como alvo para um \textit{extended 
tracking} (rastreamento extendido, em português), um jogador que deseja utilizar 
o ARCGP deverá fazer o \textit{download}, imprimir e posicionar essa imagem onde 
ele deseja que seja o campo de jogo.

A partir de então, a aplicação será desenvolvida de acordo com os requisitos 
explanados na sessão \ref{objetivo}.

No decorrer e após essa etapa, serão feitos testes com dois jogadores para 
verificar a funcionalidade da ferramenta.

Por fim, a ferramenta será apresentada à comunidade através de um novo artigo.

\section{Resultados Esperados}
\label{resultados_esperados}
Espera-se que, ao final do projeto, esteja desenvolvida uma aplicação capaz de 
auxiliar dois jogadores a jogar um jogo de cartas mesmo que estejam fisicamente 
distantes um do outro. A experiência do jogador deverá lembrar, o máximo 
possível, sua equivalente no mundo real.

\section{Cronograma}
\label{cronograma}
De 23 de setembro de 2015 a 1 de Outubro de 2015, criação de banco de dados 
com pelo menos dois decks completos de cartas para testes e seleção de uma 
figura adequada para o rastreamento extendido da mesa de jogo.
De 23 de Setembro de 2015 a 8 de Outubro de 2015, estudo de Unity, Vuforia e 
C\#.
De 8 de Outubro de 2015 a 22 de Outubro de 2015, desenvolvimento da aplicação 
ARCGP.
De 23 de Outubro de 2015 a 5 de Novembro de 2015, testes e refatoração da 
aplicação para encorporar ajustes.
De 5 de Novembro de 2015 a 17 de Novembro de 2015, escrita do artigo técnico de 
apresentação da aplicação.

\section{Conclusão}
\label{conclusao}
Neste artigo, foi descrito o projeto da Plataforma para Jogos de Cartas com 
Realidade Aumentada. Foram explicitados seus objetivos, metodologia, resultados 
esperados e o cronograma para conclusão do projeto.

Além disso, foram levantados alguns trabalhos com matérias parecidas e foram 
esclarecidas as diferentes entre eles e o ARCGP.

\bibliographystyle{IEEEtran}
\bibliography{bibi}
\end{document}