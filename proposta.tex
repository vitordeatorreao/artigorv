\documentclass[conference]{IEEEtran}
\usepackage{filecontents}
\usepackage[noadjust]{cite}
\usepackage[portuguese]{babel}
\usepackage[utf8]{inputenc}

\begin{filecontents*}{bibi.bib}
	
     @ARTICLE{lee2005tarboard,
   author = {Lee, Wonwoo and Woo, Woontack and Lee, Jongweon},
   title = {Tarboard: Tangible augmented reality system for table-top game environment},
   journal = {2nd International Workshop on Pervasive Gaming Applications, PerGamese},
   year = {2005},
   volume = {5},
   pages = {0-4}
   }
   
   @book{burdea2003virtual,
   	title={Virtual Reality Technology},
   	author={Burdea, G.C. and Coiffet, P.},
   	number={v. 1},
   	isbn={9780471360896},
   	lccn={2002038088},
   	series={Academic Search Complete},
   	year={2003},
   	publisher={Wiley}
   }
   
   @inproceedings{Lam:2006:AAR:1128923.1128987,
   	author = {Lam, Albert H. T. and Chow, Kevin C. H. and Yau, Edward H. H. and Lyu, Michael R.},
   	title = {ART: Augmented Reality Table for Interactive Trading Card Game},
   	booktitle = {Proceedings of the 2006 ACM International Conference on Virtual Reality Continuum and Its Applications},
   	series = {VRCIA '06},
   	year = {2006},
   	isbn = {1-59593-324-7},
   	location = {Hong Kong, China},
   	pages = {357--360},
   	numpages = {4},
   	url = {http://doi.acm.org/10.1145/1128923.1128987},
   	doi = {10.1145/1128923.1128987},
   	acmid = {1128987},
   	publisher = {ACM},
   	address = {New York, NY, USA},
   	keywords = {augmented reality, card game environment, computer entertainment},
   } 
   
   @misc{magic,
   	title = {Magic: The Gathering},
   	howpublished = {\url{http://magic.wizards.com/pt-br}},
   	note = {Accessed: 2015-09-15}
   }
   
   @misc{magic_duels,
   	title = {Magic Duels},
   	howpublished = {\url{http://magic.wizards.com/content/magic-duels}},
   	note = {Accessed: 2015-09-15}
   }
   
   @misc{unity,
   	title = {Unity - Game Engine},
   	howpublished = {\url{http://unity3d.com/pt/}},
   	note = {Accessed: 2015-09-16}
   }
   
   @misc{vuforia,
   	title = {Vuforia Developer Portal},
   	howpublished = {\url{https://developer.vuforia.com/}},
   	note = {Accessed: 2015-09-16}
   }
\end{filecontents*}

\hyphenation{op-tical net-works semi-conduc-tor}    

\title{Jogo de Cartas Remoto em Realidade Aumentada}

\author{Vítor de Albuquerque Torreão}


\markboth{Disciplina de Realidade Virtual, Setembro~2015}%
{Shell \MakeLowercase{\textit{et al.}}: Bare Demo of IEEEtran.cls for Journals}

\begin{document}
\maketitle

\begin{abstract}
	Os jogos do mundo real e virtual possuem suas próprias vantagens distintas. 
	A realidade aumentada nos permite combinar o melhor dos dois mundos e criar 
	novas formas de jogar. Esta proposta pretende introduzir o ARCGP (Plataforma 
	para jogos de cartas com realidade aumentada). A ideia é prover aos 
	jogadores a experiência dos jogos originais, porém com a possibilidade de 
	jogar remotamente, sem ter de sentar à frente do computador e com baixo 
	custo, sem necessidade de câmeras, mesas vidradas, espelho ou televisores. 
	A plataforma requer apenas um dispositivo capaz de rodar uma aplicação 
	Unity, seja um smartphone ou computador pessoal.
\end{abstract}

\begin{IEEEkeywords}
	Realidade Aumentada, Entretenimento.
\end{IEEEkeywords}

\section{Introdução}
Realidade misturada combina os conteúdos do mundo real com a imaginação virtual.
A realidade aumentada (AR) é um subconjunto da realidade misturada, onde o 
conteúdo digital são sobrepostos aos objetos reais do mundo. Aplicações de 
realidade aumentada suplementam o mundo real com objetos virtuais, de forma que 
o conteúdo gerado pelo computador é adicionado ao ambiente real de forma 
interativa e em tempo real \cite{burdea2003virtual}.

Essas propriedades da realidade aumentada provêm melhorias fascinantes para os 
jogos do mundo real, fazendo-os mais agradáveis e atrativos 
\cite{Lam:2006:AAR:1128923.1128987}. Jogos de cartas são exemplos de jogos que 
podem usufruir das vantagens da Realidade Aumentada. Jogos, como \textit{Magic: 
The Gathering} \cite{magic}, precisam de, no mínimo, dois jogadores. Mas não é 
sempre fácil reunir-se com outras pessoas para jogar, de forma que uma 
plataforma que permitisse a dois jogadores disputar uma partida de forma 
remota criaria mais oportunidades de jogo.

Já houveram implementações de \textit{Magic} para computador e dispositivos 
móveis \cite{magic_duels}. No entanto, as plataformas nos quais essas 
aplicações estão disponíveis quebram  a familiaridade do jogador com a forma 
de jogar com a qual está acostumado.

A plataforma hipotética deveria então, manter elementos das partidas de forma 
que o jogador tivesse a impressão de estar jogando o mesmo jogo, da mesma forma.

Neste artigo será proposta a plataforma ARCGP (\textit{Augmented Reality Card 
Game Platform}) para jogos de cartas. A plataforma consistirá em uma aplicação 
Unity \cite{unity} utilizando a API do Vuforia \cite{vuforia}. 

Este artigo está estruturado da seguinte forma: na sessão \ref{rel} serão 
apresentados trabalhos relacionados; na sessão \ref{plataforma}, será 
exposta a plataforma, tal como foi idealizada; na sessão \ref{metodologia} será 
apresentada a metodologia a ser utilizada no desenvolvimento da plataforma; na 
sessão \ref{cronograma} será descrito o cronograma a ser seguido para o 
desenvolvimento; por fim, a sessão \ref{conclusao} trás algumas considerações 
finais sobre o projeto.

\section{Trabalhos Relacionados}
\label{rel}
Aqui vão os trabalhos relacionados

\section{Plataforma}
\label{plataforma}
Aqui vai sobre a plataforma

\section{Metodologia}
\label{metodologia}
Aqui vai a metodologia

\section{Cronograma}
\label{cronograma}
Aqui vai o cronograma

\section{Conclusão}
\label{conclusao}
The conclusion goes here.

\bibliographystyle{IEEEtran}
\bibliography{bibi}
\end{document}